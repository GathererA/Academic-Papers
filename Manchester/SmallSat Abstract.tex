\documentclass{article}

\usepackage{gensymb}


\begin{document}

\title{Trajectory Optimization for Magnetorquer-Based Underactuated Control of Small Satellites}
\author{Andrew Gatherer, Zac Manchester}
\date{August 2019}
\maketitle

\begin{abstract}
Given the ever-decreasing size and ever-increasing scope of small satellites, novel methods of precise satellite attitude determination are now attractive and technically viable. Interactions between on-board magnetorquers and the Earth's magnetic field have been used for decades on satellites to work in concert with momentum wheel and control moment gyroscopes to precisely alter the attitude of Earth-orbiting bodies. With extremely small satellites, however, mechanical actuation devices become exorbitantly expensive to operate, both in mission cost and in power required. \\
\indent This paper presents a magnetorquer-only control method that utilizes trajectory optimization techniques to circumvent the under-actuated nature of satellite magnetic field interactions. Given a known orbital trajectory and desired attitude states, the model utilizes simplified dynamics and a trajectory optimization package with iLQR and augmented Lagrangian techniques to arrive at a nominal input profile in terms of magnetic moment for the satellite's magnetorquers. The computational complexity of this process is discussed in regards to the possible implementation on-board small satellite microprocessors. \\
\indent The input profile is then wrapped in a time varying LQR control system and simulated in environment with randomized noise to estimate the validity of the given input with a given estimation error. The final simulation is carried out in an environment that includes an IGRF magnetic field, nrlmsise-00 atmosphere model, and solar radiation, whereas the input profile was developed in an environment that simply included an IGRF magnetic field.\\
\indent The combined optimal trajectory and time varying LQR control scheme demonstrated significant control authority and robustness, enabling significant slewing of satellites regardless of orbital position or orbital parameters. As an example case, a generic 1U cubesat was tested with commercially available compact magnetorquers. The control scheme allowed the satellite to rotate 90$\degree$ within 5$\%$-10$\%$ of an inclined low earth orbit using a fraction of the available magnetic moment. This degree of controllability has wide applications, from jitter-free telescope control to efficient attitude stabilization.  

\end{abstract}

\section{Contact Information}
\begin{tabular}{l | c | c | r}
	\hline
	Title & Name & Affiliation & Contact\\
	\hline
	Point of Contact & Zac Manchester & Stanford University & zacmanchester@stanford.edu\\
	Presenter & Andrew Gatherer & Stanford University & gatherer@stanford.edu\\
\end{tabular}
	
\section{Additional Information}
This abstract is intended for the Advanced Technologies Session Topic. Please consider this abstract for the Pre-Conference Workshop and Poster Sessions. Additionally, this paper will also be submitted to the Student Competition. Thank you. 

\end{document}